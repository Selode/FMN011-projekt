\documentclass{article}
\usepackage[utf8]{inputenc}
\usepackage{listings}
\usepackage{xcolor}
\usepackage{textcomp}

\usepackage{graphicx}

\usepackage{hyperref}


\title{Numerisk Analys - Proj 1}
\author{Sebastian Karabeleski\\940228-0375\\dat12sk1@student.lu.se \and Martin Bergman\\950720-3314\\dat14mb1@student.lu.se}
\date{}

\begin{document}

\maketitle

\section{Introduction and Problem background}
In this project we tried to simulate a Stewart platform, using numerical methods. A Stewart platform is a type of parallel robot that has six prismatic actuators, commonly hydraulic jacks or electric actuators, attached in pairs to three positions on the platform's baseplate, crossing over to three mounting points on a top plate. 


\section{Numerical considerations}
To achieve our goal we used the programming language Python. There, we constructed one program containing methods for calculating the formulas presented by the instructions.
To check our answers we utilized Python's/scipy's method \emph{fsolve}. For plotting our results we used the \emph{matplotlib} library.

\section{Results}
\subsection{Task 1}
When we called the fuction \emph{geval} with the length three(\emph{L = 3}), we got a \textbf{math domain error}. With a length of 8 we get the height:
\[[2.7838, 2.7838, 2.7838 ]\]

\section{Analysis}
Our \textbf{math domain error} when we tried the length three appeared because we were trying to take the square root of a negative number.
Since we used Python, which is zero-indexed, we had to change some of the formulas. 

\section{Lessons learned}

\section{Acknowledgements}
\end{document}
